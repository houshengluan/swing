\documentclass{article}
\usepackage{ctex}
\usepackage[colorlinks,linkcolor=blue,bookmarksnumbered]{hyperref}

\begin{document}
\title{Python学习笔记}
\date{}
\author{}
\maketitle

\section{引言}
现在常用的有python2和python3两个版本,语法有所不同。现在先学习python3。学习网站包括
\begin{itemize}
\item http://www.runoob.com/python3/python3-basic-syntax.html
\end{itemize}

\section{基础语法}
\subsection{声明}
\begin{center}
\#!/usr/bin/python3\\
\# -*- coding: utf-8 -*-
\end{center}
第一行表示,在Linux/Unix系统中,可以让Python脚本可以像SHELL脚本一样直接执行(通过./XXX.py命令)。
第二行表示文件的编码。

\subsection{标识符}
\begin{itemize}
\item 第一个字符必须是字母表中字母或下划线$\_$。
\item 标识符的其他的部分由字母、数字和下划线组成。
\item 标识符对\textbf{大小写敏感}。
\end{itemize}

\subsection{注释}
单行注释以“$\#$”开始,多行注释可以用'{}'{}' 和 "{}"{}"。例如:
\begin{table}[htp]
\centering
\begin{tabular}{|l|}
\hline
\# 第一个注释\\
\# 第二个注释\\
\\
'{}'{}'\\
第三注释\\
第四注释\\
'{}'{}'\\
\\
"{}"{}"\\
第五注释\\
第六注释\\
"{}"{}"\\\hline
\end{tabular}
\end{table}
\subsection{缩进}
python最具特色的就是使用缩进来表示代码块,不需要使用大括号 \{\} 。

缩进的空格数是可变的,但是同一个代码块的语句必须包含相同的缩进空格数。
\subsection{多行语句}
对于一行写不完的长语句,我们可以使用反斜杠($\backslash$)来实现多行语句。例如:
\begin{center}
total = item$\_$one + $\backslash$ \\
        item$\_$two
\end{center}
但是在 [], \{\}, 或 () 中的多行语句,不需要使用反斜杠($\backslash$)。

\subsection{同一行显示多条语句}
Python可以在同一行中使用多条语句,语句之间使用分号(;)分割。

\subsection{数据类型}
\subsubsection{变量}
变量不需要声明,但每个变量在使用前都需要赋值,赋值后的变量才会被创建。

多变量赋值:a = b = c = 1 或 a, b, c = 1, 2, "runoob"

一个变量可以通过赋值指向不同类型的对象。


\subsubsection{标准数据类型}
Python3 中有六个标准的数据类型:
\begin{itemize}
\item Number(数字)
\item String(字符串)
\item List(列表)
\item Tuple(元组)
\item Sets(集合)
\item Dictionary(字典)
\end{itemize}

其中,只有 List(列表)和 Dictionary(字典)可变,其余都不可变。  

布尔类型在数字类型中。

\subsubsection{数据类型判断}
Python中的变量没有类型,我们所说的"类型"是变量所指的内存中对象的类型。可以通过内置的 type() 函数可以用来查询变量所指的对象类型,例如"a=1; type(a)"。此外还可以用 isinstance 来判断某个变量在内存中对象的类型。例如"a=1; isinstance(a, int)"

二者的区别在于:type()不会认为子类是一种父类类型。isinstance()会认为子类是一种父类类型。

\subsubsection{数字(Number)类型}
python中数字有四种类型:整数、布尔型、浮点数和复数。
\begin{itemize}
\item int (整数), 如 1, 只有一种整数类型 int,表示为长整型,没有 python2 中的 Long。
\item bool (布尔), 如 True。
\item float (浮点数), 如 1.23、3E-2
\item complex (复数), 如 1 + 2j、 1.1 + 2.2j
\end{itemize}

数值的除法包含两个运算符:/ 返回一个浮点数,// 返回一个整数。
\subsubsection{字符串(String)类型}
\begin{itemize}
\item python中单引号和双引号使用完全相同。
\item 字符串可以用 + 运算符连接在一起,用 * 运算符重复(例如print(str * 2) 表示输出字符串两次)。
\item Python 中的字符串有两种索引方式,从左往右以 0 开始,从右往左以 -1 开始。例如:
\begin{itemize}
\item word[2]表示左边数第3个字符。
\item word[-3]表示右边数第3个字符。
\item word[1:3]左边数第二到第三个字符。
\item word[:2]前2个字符。
item word[3:]从第4个开始的所有后面字符。
\end{itemize}
\item Python中的字符串不能改变。
\item Python 没有单独的字符类型,一个字符就是长度为 1 的字符串。
字符串的截取的语法格式如下:变量[头下标:尾下标]
\end{itemize}

在 Python2 中是没有布尔型的,它用数字 0 表示 False,用 1 表示 True。到 Python3 中,把 True 和 False 定义成关键字了,但它们的值还是 1 和 0,它们可以和数字相加。

\subsubsection{列表(List)类型}
列表中元素的类型可以不相同,它支持数字,字符串甚至可以包含列表(所谓嵌套)(即可以定义矩阵等)。

列表是写在方括号([])之间、用逗号分隔开的元素列表。

和字符串一样,列表同样可以被索引和截取,列表被截取后返回一个包含所需元素的新列表。 

加号(+)是列表连接运算符,星号(*)是重复操作。

List中的元素是可以改变的。

\subsubsection{元组(Tuple)类型}
元组(tuple)与列表类似,不同之处在于元组的元素不能修改。元组写在小括号 () 里,元素之间用逗号隔开。 

 构造包含 0 个或 1 个元素的元组比较特殊,所以有一些额外的语法规则:

tup1 = ()    \# 空元组

tup2 = (20,) \# 一个元素,需要在元素后添加逗号

\begin{itemize}
\item 与字符串一样,元组的元素不能修改。
\item 元组也可以被索引和切片,方法一样。
\item 注意构造包含0或1个元素的元组的特殊语法规则。
\item 元组也可以使用+操作符进行拼接。
\end{itemize}

\subsubsection{集合(Set)类型}
集合(set)是一个无序不重复元素的序列。

基本功能是进行成员关系测试和删除重复元素。

可以使用大括号\{\} 或者 set() 函数创建集合,注意:创建一个空集合必须用 set() 而不是 \{\},因为 \{\} 是用来创建一个空字典。 
\subsubsection{字典(Dictionary)类型}
字典(dictionary)是Python中另一个非常有用的内置数据类型。

列表是有序的对象集合,字典是无序的对象集合。两者之间的区别在于:字典当中的元素是通过键来存取的,而不是通过偏移存取。

字典是一种映射类型,字典用"\{\}"标识,它是一个无序的键(key) : 值(value)对集合。

键(key)必须使用不可变类型。

在同一个字典中,键(key)必须是唯一的。 

另外,字典类型也有一些内置的函数,例如clear()、keys()、values()等。

注意:

    1、字典是一种映射类型,它的元素是键值对。

    2、字典的关键字必须为不可变类型,且不能重复。

    3、创建空字典使用 \{ \}。

\subsubsection{序列}
string、list和tuple都属于sequence(序列)。

\subsubsection{数据类型之间的转换}
\begin{itemize}
\item int(x), 转换为int
\item float(x), 转换为float
\item str(x), 转换为字符型
\item tuple(x), 将序列x转换为tuple
\item list(x), 将序列x转换为列表
\item set(x), 将序列x转换为set
\item dict(d) 创建一个字典。d 必须是一个序列 (key,value)元组。
\item chr(x) 将一个整数转换为一个字符
\item ord(x) 将一个字符转换为它的整数值
\item hex(x) 将一个整数转换为一个十六进制字符串
\item oct(x) 将一个整数转换为一个八进制字符串
\end{itemize}

\subsection{空行}
函数之间或类的方法之间用空行分隔,表示一段新的代码的开始。

书写时不插入空行,Python解释器运行也不会出错。但是空行的作用在于分隔两段不同功能或含义的代码,便于日后代码的维护或重构。

\subsection{控制台输入输出}
\subsubsection{控制台输入}
a=input("请输入:")
\subsubsection{控制台输出}
print("输出")

print 默认输出是换行的,如果要实现不换行需要在变量末尾加上 end=""


\subsection{import 与 from...import}
在 python 用 import 或者 from...import 来导入相应的模块。

将整个模块(somemodule)导入,格式为: import somemodule

从某个模块中导入某个函数,格式为: from somemodule import somefunction

从某个模块中导入多个函数,格式为: from somemodule import firstfunc, secondfunc, thirdfunc

将某个模块中的全部函数导入,格式为: from somemodule import *

\subsection{运算符}

















\end{document}

